%% abtex2-modelo-trabalho-academico.tex, v-1.7.1 laurocesar
%% Copyright 2012-2013 by abnTeX2 group at http://abntex2.googlecode.com/ 
%%
%% This work may be distributed and/or modified under the
%% conditions of the LaTeX Project Public License, either version 1.3
%% of this license or (at your option) any later version.
%% The latest version of this license is in
%%   http://www.latex-project.org/lppl.txt
%% and version 1.3 or later is part of all distributions of LaTeX
%% version 2005/12/01 or later.
%%
%% This work has the LPPL maintenance status `maintained'.
%% 
%% The Current Maintainer of this work is the abnTeX2 team, led
%% by Lauro César Araujo. Further information are available on 
%% http://abntex2.googlecode.com/
%%
%% This work consists of the files abntex2-modelo-trabalho-academico.tex,
%% abntex2-modelo-include-comandos and abntex2-modelo-references.bib
%%

% ------------------------------------------------------------------------
% ------------------------------------------------------------------------
% abnTeX2: Modelo de Trabalho Academico (tese de doutorado, dissertacao de
% mestrado e trabalhos monograficos em geral) em conformidade com 
% ABNT NBR 14724:2011: Informacao e documentacao - Trabalhos academicos -
% Apresentacao
% ------------------------------------------------------------------------
% ------------------------------------------------------------------------

\documentclass[
	% -- opções da classe memoir --
	12pt,			% tamanho da fonte
	openright,		% capítulos começam em pág ímpar (insere página vazia caso preciso)
	oneside,			% para impressão em verso e anverso. Oposto a oneside
	a4paper,			% tamanho do papel. 
	% -- opções da classe abntex2 --
	chapter=TITLE,		% títulos de capítulos convertidos em letras maiúsculas
	%section=TITLE,		% títulos de seções convertidos em letras maiúsculas
	%subsection=TITLE,	% títulos de subseções convertidos em letras maiúsculas
	%subsubsection=TITLE,% títulos de subsubseções convertidos em letras maiúsculas
	% -- opções do pacote babel --
	english,			% idioma adicional para hifenização
%	spanish,			% idioma adicional para hifenização
	brazil,			% o último idioma é o principal do documento
	]{abntex2}


% ---
% PACOTES
% ---

% ---
% 2Pacotes fundamentais 
% ---
\usepackage[brazil]{babel}
\selectlanguage{brazil}
\usepackage{cmap}			% Mapear caracteres especiais no PDF
\usepackage{lmodern}			% Usa a fonte Latin Modern			
\usepackage[T1]{fontenc}		% Selecao de codigos de fonte.
\usepackage[utf8]{inputenc}	% Codificacao do documento (conversão automática dos acentos)
\usepackage{lastpage}			% Usado pela Ficha catalográfica
\usepackage{indentfirst}		% Indenta o primeiro parágrafo de cada seção.
\usepackage{color}			% Controle das cores
\usepackage{graphicx}			% Inclusão de gráficos
% ---


% ---
% Pacotes de citações
% ---
\usepackage[brazilian,hyperpageref]{backref}	 % Paginas com as citações na bibl
\usepackage[alf]{abntex2cite}				% Citações padrão ABNT

% --- 
% CONFIGURAÇÕES DE PACOTES
% --- 

% ---
% Configurações do pacote backref
% Usado sem a opção hyperpageref de backref
\renewcommand{\backrefpagesname}{Citado na(s) página(s):~}
% Texto padrão antes do número das páginas
\renewcommand{\backref}{}
% Define os textos da citação
\renewcommand*{\backrefalt}[4]{
	\ifcase #1 %
		Nenhuma citação no texto.%
	\or
		Citado na página #2.%
	\else
		Citado #1 vezes nas páginas #2.%
	\fi}%
% ---


% ---
% Informações de dados para CAPA e FOLHA DE ROSTO
% ---
\titulo{DESENVOLVIMENTO DE TECNOLOGIA FOCADO EM IHC}

\autor{
  Guilherme Anderzen Polido
}
\local{TOLEDO, PR}
\data{2019}

\orientador{Eduardo Pezutti Beletato dos Santos}

\instituicao{
  UNIVERSIDADE TECNOLÓGICA FEDERAL DO PARANÁ \par
  CÂMPUS TOLEDO \par
  COORDENAÇÃO DO CURSO DE TECNOLOGIA EM SISTEMAS PARA INTERNET
%  \par
%  Design Digital
%  \par
%  Programa de Graduação
  }
\tipotrabalho{Trabalho de conclusão de curso}
% O preambulo deve conter o tipo do trabalho, o objetivo, 
% o nome da instituição e a área de concentração 
\preambulo{Trabalho de Conclusão de Curso de Graduação, apresentado ao Curso Superior de Tecnologia em Sistemas para Internet, da Universidade Tecnológica Federal do Paraná – UTFPR, como requisito parcial para obtenção do título de Tecnólogo. Orientador: Prof. PROFESSOR}
% ---


% ---
% Configurações de aparência do PDF final

% alterando o aspecto da cor azul
%\definecolor{blue}{RGB}{41,5,195}
\definecolor{blue}{RGB}{0,0,0}

% informações do PDF
\makeatletter
\hypersetup{
     	%pagebackref=true,
	pdftitle={\@title}, 
	pdfauthor={\@author},
    	pdfsubject={\imprimirpreambulo},
	pdfcreator={LaTeX with abnTeX2},
	pdfkeywords={abnt}{latex}{abntex}{abntex2}{trabalho acadêmico}, 
	colorlinks=true,       	% false: boxed links; true: colored links
    	linkcolor=blue,          	% color of internal links
    	citecolor=blue,        	% color of links to bibliography
    	filecolor=magenta,      	% color of file links
	urlcolor=blue,
	bookmarksdepth=4
}
\makeatother
% --- 

% --- 
% Espaçamentos entre linhas e parágrafos 
% --- 

% O tamanho do parágrafo é dado por:
\setlength{\parindent}{1.3cm}

% Controle do espaçamento entre um parágrafo e outro:
\setlength{\parskip}{0.2cm}  % tente também \onelineskip

% ---
% compila o indice
% ---
\makeindex
% ---

% ----
% Início do documento
% ----
\begin{document}

% Retira espaço extra obsoleto entre as frases.
\frenchspacing 

% ----------------------------------------------------------
% ELEMENTOS PRÉ-TEXTUAIS
% ----------------------------------------------------------
% \pretextual

% ---
% Capa
% ---
\imprimircapa
% ---

% ---
% Folha de rosto
% (o * indica que haverá a ficha bibliográfica)
% ---
\imprimirfolhaderosto*
% ---

% ---
% Inserir a ficha bibliografica
% ---

% Isto é um exemplo de Ficha Catalográfica, ou ``Dados internacionais de
% catalogação-na-publicação''. Você pode utilizar este modelo como referência. 
% Porém, provavelmente a biblioteca da sua universidade lhe fornecerá um PDF
% com a ficha catalográfica definitiva após a defesa do trabalho. Quando estiver
% com o documento, salve-o como PDF no diretório do seu projeto e substitua todo
% o conteúdo de implementação deste arquivo pelo comando abaixo:

% \begin{fichacatalografica}
%     \includepdf{fig_ficha_catalografica.pdf}
% \end{fichacatalografica}

%\begin{fichacatalografica}
%	\vspace*{\fill}					% Posição vertical
%	\hrule						% Linha horizontal
%	\begin{center}				% Minipage Centralizado
%	\begin{minipage}[c]{12.5cm}	% Largura
%	
%	\imprimirautor
%	
%	\hspace{0.5cm} \imprimirtitulo  / \imprimirautor. --
%	\imprimirlocal, \imprimirdata-
%	
%	\hspace{0.5cm} \pageref{LastPage} p. : il. (algumas color.) ; 30 cm.\\
%	
%	\hspace{0.5cm} \imprimirorientadorRotulo~\imprimirorientador\\
%	
%	\hspace{0.5cm}
%	\parbox[t]{\textwidth}{\imprimirtipotrabalho~--~\imprimirinstituicao,
%	\imprimirdata.}\\
%	
%	\hspace{0.5cm}
%		1. Palavra-chave1.
%		2. Palavra-chave2.
%		I. Orientador.
%		II. Universidade xxx.
%		III. Faculdade de xxx.
%		IV. Título\\ 			
%	
%	\hspace{8.75cm} CDU 02:141:005.7\\
%	
%	\end{minipage}
%	\end{center}
%	\hrule
%\end{fichacatalografica}
% ---

% ---
% Inserir errata
% ---
%%\begin{errata}
%%Elemento opcional da \citeonline[4.2.1.2]{NBR14724:2011}. Exemplo:
%%
%%\vspace{\onelineskip}
%%
%%FERRIGNO, C. R. A. \textbf{Tratamento de neoplasias ósseas apendiculares com
%%reimplantação de enxerto ósseo autólogo autoclavado associado ao plasma
%%rico em plaquetas}: estudo crítico na cirurgia de preservação de membro em
%%cães. 2011. 128 f. Tese (Livre-Docência) - Faculdade de Medicina Veterinária e
%%Zootecnia, Universidade de São Paulo, São Paulo, 2011.
%%
%%\begin{table}[htb]
%%\center
%%\footnotesize
%%\begin{tabular}{|p{1.4cm}|p{1cm}|p{3cm}|p{3cm}|}
%%  \hline
%%   \textbf{Folha} & \textbf{Linha}  & \textbf{Onde se lê}  & \textbf{Leia-se}  \\
%%    \hline
%%    1 & 10 & auto-conclavo & autoconclavo\\
%%   \hline
%%\end{tabular}
%%\end{table}
%%
%%\end{errata}
% ---

% ---
% Inserir folha de aprovação
% ---

% Isto é um exemplo de Folha de aprovação, elemento obrigatório da NBR
% 14724/2011 (seção 4.2.1.3). Você pode utilizar este modelo até a aprovação
% do trabalho. Após isso, substitua todo o conteúdo deste arquivo por uma
% imagem da página assinada pela banca com o comando abaixo:
%
%
%\begin{folhadeaprovacao}
%
%  \begin{center}
%    {\ABNTEXchapterfont\large\imprimirautor}
%
%    \vspace*{\fill}\vspace*{\fill}
%    {\ABNTEXchapterfont\bfseries\Large\imprimirtitulo}
%    \vspace*{\fill}
%    
%    \hspace{.45\textwidth}
%
%
%%{\large COMISSÃO JULGADORA}
%%
%%{\large MONOGRAFIA PARA A CONCLUSÃO DO CURSO\\DE DESIGN DIGITAL}
%    
%
%   \assinatura{\textbf{\imprimirorientador} \\ Presidente e Orientador} 
%%   \assinatura{\textbf{Professor} \\ 2º Examinador}
%%   \assinatura{\textbf{Professor} \\ 3º Examinador}
%
%
%
%      
%   \begin{center}
%    \vspace*{0.5cm}
%    {\large Araraquara,  \today}
%    \vspace*{1cm}
%  \end{center}
%  
%\end{folhadeaprovacao}
% ---

% ---
% Dedicatória
% ---
%%\begin{dedicatoria}
%%   \vspace*{\fill}
%%   \centering
%%   \noindent
%%   \textit{ Este trabalho é dedicado às crianças adultas que,\\
%%   quando pequenas, sonharam em se tornar cientistas.} \vspace*{\fill}
%%\end{dedicatoria}
% ---

% ---
% Agradecimentos
% ---
%\begin{agradecimentos}
%
%
%
%
%\end{agradecimentos}
% ---

% ---
% Epígrafe
% ---
\begin{epigrafe}
    \vspace*{\fill}
	\begin{flushright}
		\textit{``Não vos amoldeis às estruturas deste mundo, \\
		mas transformai-vos pela renovação da mente, \\
		a fim de distinguir qual é a vontade de Deus: \\
		o que é bom, o que Lhe é agradável, o que é perfeito.\\
		(Bíblia Sagrada, Romanos 12, 2)}
	\end{flushright}
\end{epigrafe}
% ---

% ---
% RESUMOS
% ---

% resumo em português
\begin{resumo}
Escrever o resumo

 \vspace{\onelineskip}
    
 \noindent
 \textbf{Palavras-chaves}: .
\end{resumo}

% resumo em inglês
\begin{resumo}[Abstract]
 \begin{otherlanguage*}{english}
Write Abstract

   \vspace{\onelineskip}
 
   \noindent 
   \textbf{Key-words}: .
 \end{otherlanguage*}
\end{resumo}
% ---

% ---
% inserir lista de ilustrações
% ---
\pdfbookmark[0]{\listfigurename}{lof}
\listoffigures*
\cleardoublepage
% ---

% ---
% inserir lista de tabelas
% ---
\pdfbookmark[0]{\listtablename}{lot}
\listoftables*
\cleardoublepage
% ---

% ---
% inserir lista de abreviaturas e siglas
% ---
%%%\begin{siglas}
%%%  \item[Fig.] Area of the $i^{th}$ component
%%%  \item[456] Isto é um número
%%%  \item[123] Isto é outro número
%%%  \item[lauro cesar] este é o meu nome
%%%\end{siglas}
% ---

% ---
% inserir lista de símbolos
% ---
%%%\begin{simbolos}
%%%  \item[$ \Gamma $] Letra grega Gama
%%%  \item[$ \Lambda $] Lambda
%%%  \item[$ \zeta $] Letra grega minúscula zeta
%%%  \item[$ \in $] Pertence
%%%\end{simbolos}
% ---

% ---
% inserir o sumario
% ---
\pdfbookmark[0]{\contentsname}{toc}
\tableofcontents*
\cleardoublepage
% ---



% ----------------------------------------------------------
% ELEMENTOS TEXTUAIS
% ----------------------------------------------------------
\textual

% ----------------------------------------------------------
% Introdução
% ----------------------------------------------------------
\chapter[Introdução]{Introdução}

Na atualidade o sedentarismo, bem como outros problemas diretamente relacionados à falta de exercícios físicos, como males coronários, insuficiência respiratória, entre outros, têm levado mais e mais pessoas a procurar as academias de ginástica no intuito de desfrutarem de uma melhor qualidade de vida, objetivando que seu cotidiano passa a ter melhor saúde. 

Assim, aponta-se a importância das academias de ginástica para o bem estar dos seus frequentadores. Contudo, evidencia-se a importância de tais empresas valerem-se de modernas tecnologias para alcançarem uma melhoria na eficiência das atividades, elevando o número de clientes e, desta forma, estenderem seus benefícios a uma parcela maior da população.

\section{Objetivos}
\label{obj}

O principal objetivo deste trabalho é a elaboração de um aplicativo móvel para uso em academias de ginástica, atuando diretamente em setores como, por exemplo, a musculação. Com o desenvolvimento de tal programa pretende-se não somente favorecer à saúde da população, mas também os estabelecimentos empresariais direcionados às atividades físicas, pois, com uma pesquisa realizada em três academias, foi possível verificar que o número de pessoas iniciantes é grande e a maioria delas possuem um quadro de instrutores baixo, com o aplicativo em estudo por este trabalho pretende auxiliar os iniciantes demonstrando através de vídeo aulas a execução dos exercícios. 



% ---
% Capitulo de revisão de literatura
% ---
\chapter{Referencial Teórico}
\label{refteo}

O \textbf{referencial} teórico se dará através da leitura de livros e artigos publicados em revistas, jornais e disponibilizados em sites da Internet, com a consulta a autores que abordam o tema proposto para este trabalho.



\chapter{Metodologia}
\label{met}




\chapter{Desenvolvimento}
\label{des}
% Testos
dadsasad
dadsasadsada


\begin{figure}[h!]
  \centering
		\includegraphics[width=0.8\textwidth]{./img/img1.jpg}
		\caption{A picture of a gull.}
		\label{fig:f1} %para usar na referencia
\end{figure}

\begin{table}[]
	\begin{tabular}{ll}
	\textbf{weqewqeqewq} & \textbf{dsadsadas} \\
	ewqeqw               & 2                  \\
	dasd                 & 3                  \\
	dsadas               & 4                 
	\end{tabular}
	\end{table}


\chapter{Conclusão}

% ---
% Finaliza a parte no bookmark do PDF, para que se inicie o bookmark na raiz
% ---
\bookmarksetup{startatroot}% 
% ---






% ----------------------------------------------------------
% ELEMENTOS PÓS-TEXTUAIS
% ----------------------------------------------------------
\postextual


% ----------------------------------------------------------
% Referências bibliográficas
% ----------------------------------------------------------
%\bibliography{abntex2-modelo-references}

%\bibliography{biblio}


% ----------------------------------------------------------
% Glossário
% ----------------------------------------------------------
%
% Consulte o manual da classe abntex2 para orientações sobre o glossário.
%
%\glossary

%% ----------------------------------------------------------
%% Apêndices
%% ----------------------------------------------------------
%
%% ---
%% Inicia os apêndices
%% ---
%\begin{apendicesenv}
%
%% Imprime uma página indicando o início dos apêndices
%\partapendices
%
%% ----------------------------------------------------------
%\chapter{Quisque libero justo}
%% ----------------------------------------------------------
%
%\lipsum[50]
%
%% ----------------------------------------------------------
%\chapter{Nullam elementum urna vel imperdiet sodales elit ipsum pharetra ligula
%ac pretium ante justo a nulla curabitur tristique arcu eu metus}
%% ----------------------------------------------------------
%\lipsum[55-57]
%
%\end{apendicesenv}
%% ---
%
%
%% ----------------------------------------------------------
%% Anexos
%% ----------------------------------------------------------


\begin{anexosenv}

% Imprime uma página indicando o início dos anexos
\partanexos



\end{anexosenv}


%---------------------------------------------------------------------
% INDICE REMISSIVO
%---------------------------------------------------------------------
\printindex

\end{document}
